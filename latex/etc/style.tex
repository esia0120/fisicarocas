% Formato del título de capítulos y secciones
\titleformat{\chapter}[block]{\filleft\normalfont\huge\bfseries}{\thechapter.}{.5em}{\Huge}[\vspace{2pt}{\titlerule[2pt]}]

\titlespacing*{\chapter}{0pt}{-19pt}{25pt}

\titleformat{\section}[block]{\normalfont\Large\bfseries}{\thesection.}{.5em}{\Large}

\titleformat{\part}[block]{\titlerule[2pt]\normalfont\Huge\bfseries\centering}{Parte \Roman{part}\\\vspace{15pt}}{0pt}{\Huge}[\vspace{2pt}{\titlerule[2pt]}]

% Tamaños y estilos de elementos en la TOC
\DeclareTOCStyleEntry[
    linefill=\bfseries\TOCLineLeaderFill,
    beforeskip=12pt,
    entrynumberformat=\chapterprefixintoc,
    entryformat=\chaptertocformat,
    pagenumberformat=\chaptertocformat,
    dynnumwidth
]{tocline}{chapter}

\DeclareTOCStyleEntry[
    % linefill=\bfseries\TOCLineLeaderFill,
    beforeskip=30pt,
    entrynumberformat=\chapterprefixintoc,
    entryformat=\parttocformat,
    pagenumberformat=\partpagetocformat,
    numwidth=0pt
]{tocline}{part}

\newcommand\chaptertocformat[1]{\large{\textbf{#1}}}%
\newcommand\chapterprefixintoc[1]{#1}%
\newcommand\parttocformat[1]{\Large{\textbf{#1}}}%
\newcommand\partpagetocformat[1]{} % Don't print the page number for parts

% Alias para estilos de texto comunes
\newcommand{\negritas}[1]{\textbf{#1}}
\newcommand{\cursiva}[1]{\textit{#1}}
\newcommand{\codigo}[1]{\texttt{#1}}

% Formato del código fuente con lstlisting
\lstset{
  basicstyle=\ttfamily,
  breaklines=true,
}

% Márgenes
\geometry{
    a4paper,
    margin=2.75cm
}
\setlength{\marginparwidth}{2cm} 

% Limite de profundidad del índice
\setcounter{tocdepth}{2}

% Eliminar el guionado
\tolerance=1
\emergencystretch=\maxdimen
\hyphenpenalty=10000
\hbadness=10000

% Indentación de párrafos
\setlength{\parindent}{.75cm}

\renewcommand{\lstlistingname}{Extracto de código}
\renewcommand*{\lstlistlistingname}{Índice de extractos de código}

% Comandos para establecer variables
\newcommand{\setTitle}[1]{\def\tfgTitle {#1}}
\newcommand{\setAuthor}[1]{\def\tfgAuthors {#1}}
\newcommand{\setDegree}[1]{\def\tfgDegree {#1}}
\newcommand{\setSupervisor}[1]{\def\tfgSupervisor {#1}}
\newcommand{\setDepartment}[1]{\def\tfgDepartment {#1}}
\newcommand{\setMonth}[1]{\def\tfgMonth {#1}}
\newcommand{\setYear}[1]{\def\tfgYear {#1}}
\newcommand{\setDedication}[1]{\def\tfgDedication {#1}}

% Estilos para el código
% Configuración genérica
\definecolor{codegreen}{rgb}{0,0.6,0}
\definecolor{codegray}{rgb}{0.5,0.5,0.5}
\definecolor{codepurple}{rgb}{0.58,0,0.82}
\definecolor{editorOcher}{rgb}{0.8, 0.3, 0} % #FF7F00 -> rgb(239, 169, 0)
\definecolor{editorGreen}{rgb}{0, 0.5, 0} % #007C00 -> rgb(0, 124, 0)

\lstdefinestyle{listingstyle}{
    backgroundcolor=\color{white},  
    keywordstyle=\bfseries\color{blue},
    numberstyle=\tiny\color{codegray},
    stringstyle=\color{editorGreen},
    commentstyle=\color{codegray},
    basicstyle=\ttfamily\color{black},
    breakatwhitespace=false,         
    breaklines=true,                 
    captionpos=b,                    
    keepspaces=true,                 
    numbers=left,                    
    numbersep=5pt,                  
    showspaces=false,                
    showstringspaces=false,
    showtabs=false,                  
    tabsize=2,
    frame=tb,
    keywords=[2]{True,False},
    literate=%
*{0}{{{\color{editorOcher}0}}}1
{1}{{{\color{editorOcher}1}}}1
{2}{{{\color{editorOcher}2}}}1
{3}{{{\color{editorOcher}3}}}1
{4}{{{\color{editorOcher}4}}}1
{5}{{{\color{editorOcher}5}}}1
{6}{{{\color{editorOcher}6}}}1
{7}{{{\color{editorOcher}7}}}1
{8}{{{\color{editorOcher}8}}}1
{9}{{{\color{editorOcher}9}}}1,
}

\lstset{style=listingstyle}
\lstset{columns=fullflexible}

\lstdefinelanguage{css}{
  keywords={color,background-image:,margin,padding,font,weight,display,position,top,left,right,bottom,list,style,border,size,white,space,min,width, transition:, transform:, transition-property, transition-duration, transition-timing-function},	
  sensitive=true,
  morecomment=[l]{//},
  morecomment=[s]{/*}{*/},
  morestring=[b]',
  morestring=[b]",
  alsoletter={:},
  alsodigit={-}
}
% JavaScript
\lstdefinelanguage{javascript}{
  morekeywords={abstract, arguments, await, boolean, break, byte, case, catch, char, class, const, continue, debugger, default, delete, do, double, else, enum, eval, export, extends, false, final, finally, float, for, function, goto, if, implements, import, in, instanceof, int, interface, let, long, native, new, null, package, private, protected, public, return, short, static, super, switch, synchronized, this, throw, throws, transient, true, try, typeof, var, void, volatile, while, with, yield},
  morecomment=[s]{/*}{*/},
  morecomment=[l]//,
  morestring=[b]",
  morestring=[b]'
}
